\documentclass[12pt,a4paper]{report}
\usepackage[utf8]{inputenc}
\usepackage{amsmath}
\usepackage{amsfonts}
\usepackage{amssymb}
\usepackage{graphicx}
\usepackage[portuguese]{babel}
\author{Paulo Batista da Costa}
\title{PBR - Projeto Memorial : Visão do Usuário}
\begin{document}
\maketitle
\tableofcontents
\begin{quotation}
\newpage
\section{Introdução}
Início e término em: 01/04/2016

9:00 h às 12:00 h
\section{Funções de usuário}

\begin{itemize}
\item Registrar conta
\item Efetuar login
\item Recuperar senha
\item Alterar dados no sistema
\item Registrar memorial
\item Anexar Documentos
\item Obter documento HTML
\end{itemize}
\section{Questões de usuário}
\subsection{Todas as funções necessárias para escrever os cenários estão especificadas no documento de requisitos ou na especificação funcional?}
Sim. Apesar da péssima qualidade da documentação é possível abstrair os casos de uso.
\subsection{As condições iniciais para definir (iniciar) os cenários estão claras e corretas?}
Não estão todas claras. Infelizmente, a parte relacionada à interface (que interessa do ponto de vista do usuário) não foi definida.
\subsection{As interfaces entre as funções estão bem definidas e compatíveis (por ex., as entradas de uma função têm ligação com as saídas da função anterior?)}
As entradas foram bem definidas, porém há problemas quanto a definição de saídas. O documento poderia exibir as informações com mais clareza.
\subsection{Você consegue chegar num estado do sistema que deve ser evitado (por ex., por razões de segurança)?}
Não, pois no final da documentação é especificado como prioritário o fator segurança. Isto dando especificidade quanto aos cuidados da implementação. Desta forma, o usuário não poderia executar funções nas quais ele não possui uma permissão específica para tal.
\subsection{Os cenários podem fornecer diferentes respostas dependendo de como a especificação é interpretada?}
Não.
\subsection{A especificação funcional faz sentido de acordo com o que você conhece sobre essa aplicação ou sobre o que foi especificado em uma descrição geral?}
De um modo geral sim, entretanto está mal escrita.
\end{quotation}
\end{document}